\documentclass{cv}
\usepackage{xcolor}
\usepackage{multicol}
\usepackage{bibentry}
\usepackage{natbib}
\usepackage{graphicx}
\usepackage[]{fancyhdr}
\usepackage{tikz}
\RequirePackage{fontawesome}

% \usepackage{tgbonum}
% \usepackage[maxnames=99]{biblatex}
\usepackage[margin=0.8in,
  tmargin=0.5in,
  bmargin=0.5in,
  includehead,
  headsep=0.2in,
  headheight=0.3in]{geometry} % Document margins
\usepackage[colorlinks = true,
  linkcolor = blue,
  urlcolor  = magenta,
  citecolor = blue,
  anchorcolor = blue]{hyperref}
\newcommand{\tab}[1]{\hspace{.2667\textwidth}\rlap{#1}}
\newcommand{\itab}[1]{\hspace{0em}\rlap{#1}}
\newcommand{\ExternalLink}{%
    \tikz[x=1.2ex, y=1.2ex, baseline=-0.05ex]{% 
        \begin{scope}[x=1ex, y=1ex]
            \clip (-0.1,-0.1) 
                --++ (-0, 1.2) 
                --++ (0.6, 0) 
                --++ (0, -0.6) 
                --++ (0.6, 0) 
                --++ (0, -1);
            \path[draw, 
                line width = 0.5, 
                rounded corners=0.5] 
                (0,0) rectangle (1,1);
        \end{scope}
        \path[draw, line width = 0.5] (0.5, 0.5) 
            -- (1, 1);
        \path[draw, line width = 0.5] (0.6, 1) 
            -- (1, 1) -- (1, 0.6);
        }
    }

\begin{document}
  \nobibliography{references}
  \bibliographystyle{apalike}
  
  \pagestyle{fancy}
  \fancyhf{}	% removes prior settings for header and footer
  \fancyhead[L]{\fontsize{8}{8} \selectfont 
  }
  \fancyhead[R]{\fontsize{8}{8} \selectfont
  }
  \begin{center}{}
    {\textsc{\LARGE Curriculum Vitae}}\\
  \end{center}
  \vspace{0.15in}
  \hrule \vspace{0.5mm} \hrule\hrule
  \vspace{0.1in}
  

  \begin{rSection}{Personal Information}
    \begin{multicols}{2}
    \begin{tabular}{|ll}
      Full name: & Sarthak Kapoor\\
      Links: & \href[]{https://github.com/ka-sarthak}{GitHub}{}, \href[]{https://scholar.google.com/citations?user=En481soAAAAJ&hl=en}{Google Scholar}\\
      Languages: &  English (Professional proficiency), German (A1), Hindi (Native proficiency) \\
      Date of birth: &  Nov. 1, 1997 \\ 
      Nationality: &  India \\
    \end{tabular}
    \columnbreak
    \end{multicols}
  \end{rSection}

  \begin{rSection}{Education}
    \item {\bf RWTH Aachen University} \hfill {2020.11 - 2023.04}\\
    MSc Simulation Sciences\\
    Thesis: Correlative modeling of stress field with artificial neural networks
    
    \item {\bf National Institute of Technology, Warangal} \hfill {2016.08 - 2020.08} \\
    BTech Metallurgical and Materials Engineering\hfill \\
  \end{rSection}

  \begin{rSection}{Work Experience}
    \item{\bf Data Scientist} \hfill {2023.11 - Present}\\
    {\em FAIRmat, Humboldt University, Berlin}\\
    Developing data science methodologies in the field of material synthesis.

    \item {\bf Wissenschaftliche Hilfskraft} \hfill {2021.05 - 2023.09}\\
    {\em Chair for Material Mechanics, RWTH Aachen}\\
    Explored innovative ML architectures to learn non-linear computational models of physical systems. 
    Trained models (U-Net, FNO, cGAN) for stress prediction in microstructures which were 1000$\times$ faster than computation of numerical solution.
    Presented the findings at NeurIPS 2022 AI4Science workshop.
    
    \item{\bf Machine Learning Developer Intern} \hfill {2022.12 - 2023.05}\\
    {\em Ericsson Eurolab, Herzogenrath}\\
    Worked in ML-powered anomaly detection and classification system for log files. 
    Developed an interactive React-based GUI for visualizing millions of log anomaly scores, which was 100$\times$ faster than the previous version.
    Performed data exploration and developed modeling strategies for data-driven resource allocation in telecom services. 
    
    \item {\bf Application Development Analyst} \hfill {2020.09 - 2020.12}\\
    {\em Accenture Technology Center, Bengaluru}\\
    Worked in software and ML-driven solution development. 
    Participated in tech-oriented case studies and software development workshops.
    
    \item {\bf MITACS Summer Intern} \hfill {2019.05 - 2019.08}\\
    {\em Département de génie mécanique, ETS Montreal}\\
    Implemented numerical models for material flow in low-pressure metallic-powder injection molding using Finite Element methods and contributed to experimental verification.
    Designed and verified data-driven models for constitutive relations. Contributed towards a journal publication.
  \end{rSection}

  
  
  \newpage
  \begin{rSection}{Technical Projects}
    \item {\bf Phase field modeling of chemomechanical binary system} \hfill {2022.04 - 2022.08}\\
    {\em Material Mechanics, RWTH Aachen}\\
    Implemented Cahn-Hilliard and Allen-Cahn models to simulate precipitate growth dynamics under the influence of chemical and mechanical energies. 
    Used Python standard library.
    
    \item {\bf Detecting gravity waves in atmospheric temperature data} \hfill {2022.06}\\
    {\em Chair for Applied and Computational Mathematics, RWTH Aachen}\\
    Implemented an FFT-based algorithm to detect gravity waves as wave packets in 2D temperature data. 
    Lowered time complexity to O(n $\log$ n) in comparison to the previous method.
    Used Python with SciPy, JAX.
    
    \item {\bf Tracking local optima in dynamic systems}\hfill {2021.10 - 2022.02}\\
    {\em Chair for Software and Tools for Computational Engineering, RWTH Aachen}\\
    Developed local-optima-tracking software for dynamic time-evolving functions. 
    Tracked the local optima to reduce the need for global optima search from every time step to coarser time intervals. Achieved 10$\times$ lower wall time.
    Used C++ with dco/c++ library.
    
    \item {\bf Fast iterative solvers for linear systems}\hfill {2021.04 - 2021.09}\\
    {\em AICES, RWTH Aachen}\\
    Implemented multigrid solvers, Krylov-based linear system solvers (GMRES and CG) and eigensolver algorithms (Lanczos and Power Iteration). 
    Used Python standard library.
    
    
    % \item {\bf Phase field modeling of ternary system} \hfill {2018.11 - 2019.04}\\
    % {\em Metallurgical and Materials Engineering, NIT Warangal}\\
    % Developed a semi-implicit spectral PFM routine to study the growth kinetics of precipitates in a hypothetical ternary system. (C with FFT library)
  \end{rSection}
  
  \begin{rSection}{Skills} \itemsep -4pt 
    \item \textbf{Machine Learning and data analysis} \textemdash  Python with PyTorch, TensorFlow, Scikit-Learn, Pandas, NumPy, SciPy, JAX
    \item \textbf{Software development} \textemdash Python, C++, SQL, MATLAB, OpenMP, MPI, DCO
    \item \textbf{Web Development} \textemdash JavaScript, React, Jekyll
    \item \textbf{Automated CI/CD workflows} \textemdash git, Kubernetes
  \end{rSection}

  \begin{rSection}{Conferences} \itemsep -4pt 
    \item FAIRmat meets domain experts, Arnsberg, October 2023
    \item NeurIPS 2022, New Orleans, November 2022 (virtually)
    \item Machine Learning Summer School, Krakow, June 2022
  \end{rSection}
  
  \begin{rSection}{Communication}% \itemsep -4pt
    \item \bibentry{kapoor2022comparison}
    \item \bibentry{kapoor2022surrogate}
    \item \bibentry{mlss2022}
    \item \bibentry{cote2021}
  \end{rSection}
  
  \begin{rSection}{Awards} \itemsep -4pt 
    \item Full scholarship to attend Machine Learning Summer School 2022, Krakow
    \item Late Pendyala Upendra Gold Medal 2020 for academic excellence in BTech degree
    % \item Honored by Indian Institute of Metals, Hyderabad chapter, for academic excellence in BTech degree
    \item MITACS 2019 scholarship to pursue research at ÉTS Montreal \hfill 
    \item NIT Warangal Merit Scholarship (Full Tuition Award)
    \item OPJEMS award 2018 \& 2019 \hfill 
    % \item Recipient of Central Scheme National Scholarship 2016, 2017, 2018 \hfill 
  \end{rSection}
  
  \begin{rSection}{Volunteering}% \itemsep -4pt 
    \item {\bf Project Aakaar: Making geometry accessible to visually impaired}\\%\hfill {\em 2019.01 - Present}\\
    Started an assistive edu-tech project at the maker space of NIT Warangal to increase the participation of visually impaired students in STEM. Led the product design and established a strong network of designers, engineers, and researchers from MIT. \href[]{https://misti.mit.edu/humanistic-design-workshops-india-impact-and-reach}{Media coverage}.
  \end{rSection}

  \begin{rSection}{Referees}
    \item Prof. Dr. Bob Svendsen \href{https://www.cmm.rwth-aachen.de/cms/CMM/Forschung/Mitarbeiter/Leitung/~glyz/Bob-Svendsen/?allou=1}{\faExternalLink}\\
    {Chair for Material Mechanics, RWTH}\\
    {Max-Plank-Institut for Iron Research, Düsseldorf}\\
    \href{mailto:b.svendsen@mpie.de}{b.svendsen@mpie.de}

    \item Dr. Jaber Rezaei Mianroodi \href{https://www.mpie.de/person/59480/3079071}{\faExternalLink}\\
    {Ex-group leader at Max-Plank-Institut for Iron Research, Düsseldorf}\\
    {Digital R\&D, Covestro AG}\\
    \href{mailto:j.mianroodi@mpie.de}{j.mianroodi@mpie.de}
  \end{rSection}

  \end{document}
