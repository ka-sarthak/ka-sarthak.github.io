\documentclass{resume}
\usepackage[colorlinks=True]{hyperref}
\usepackage{xcolor}
\usepackage{multicol}
\usepackage{bibentry}
\usepackage{natbib}
\usepackage{graphicx}
% \usepackage[maxnames=99]{biblatex}
\usepackage[left=0.75in,top=0.6in,right=0.75in,bottom=0.6in]{geometry} % Document margins
\newcommand{\tab}[1]{\hspace{.2667\textwidth}\rlap{#1}}
\newcommand{\itab}[1]{\hspace{0em}\rlap{#1}}


\name{Sarthak Kapoor}
\begin{document}
  \nobibliography{references}
  \bibliographystyle{apalike}

  \begin{rSection}{Personal Information}\itemsep -3pt
    \begin{multicols}{2}
    \begin{tabular}{|ll}
      Address: & Kullenhofstr. 56, Aachen 52074, Germany \\
      Email: & sarthak.kapoor@rwth-aachen.de \\
      Contact: & +49 162 5483728 \\
      Languages: &  English and Hindi (Bilingual proficiency)\\
      & German (A1) \\
      Date of birth: &  Nov. 1, 1997 \\ 
      Nationality: &  India \\
      Personal Website: & \url{https://ka-sarthak.github.io/}\\
    \end{tabular}
    \columnbreak
    \hfill \includegraphics[width=3.5cm]{profile.jpg}
    \end{multicols}
  \end{rSection}

  %\begin{rSection}{Objective}
  %Self-motivated master's student seeking experience in the field of %process simulation and optimization and novel applications of Machine %Learning techniques.
  %\end{rSection}

  \begin{rSection}{Education} \itemsep -4pt
    \item {\bf RWTH Aachen University} \hfill {\em 2020.10 - Present}\\
    MSc Simulation Sciences \\
    Focus: Applied mathematics, machine learning, computational modeling 
    \item {\bf Machine Learning Summer School (MLSS\(^N\))} \hfill {\em 2022.06}\\
    {\em Hosted by Jagiellonian University, Poland}\\
    Participated with a full scholarship
    \item {\bf National Institute of Technology, Warangal} \hfill {\em 2016.08 - 2020.08} \\
    BTech Metallurgical and Materials Engineering\hfill \\
    {CGPA: 9.13/10.0} \emph{(Gold Medalist)}
    %{\bf Sacred Heart Convent School, Ludhiana} \hfill {\em 2015.03 - 2016.05}\\
    %Grade 12 CBSE (Physics, Chemistry, Mathematics, English) \hfill {Percentage: 95.4 }
    %\emph{Batch Rank: 3}
    %{\bf DAV Public School, Ludhiana} \hfill {\em 2015.03 - 2016.05}\\
    %Grade 10 CBSE \hfill {CGPA: 10}\\
  \end{rSection}

  \begin{rSection}{Work Experience}
    \item {\bf Wissenschaftliche Hilfskraft} \hfill {\em 2021.05 - Present}\\
    {\em Material Mechanics, RWTH Aachen}\\
    Developing machine-learning-based surrogate models for micromechanics simulations. \\Generative modeling using GANs, U-Net, FNO, CNNs. (Python with TensorFlow, PyTorch libraries)
    \item {\bf Application Development Analyst} \hfill {\em 2020.09 - 2020.12}\\
    {\em Accenture Technology Center, Bengaluru}
    % During my stint as an Application Development Analyst at Accenture, I learned a great deal about conceptualizing software solutions and trained in software-delivery methods.
  \end{rSection}

  \begin{rSection}{Skills}\itemsep -4pt 
    \item \textbf{Development} \textemdash  Python (TensorFlow, PyTorch, NumPy, SciPy, JAX, Scikit-Learn, Pandas), C/C++, MATLAB, Java, OpenMP, MPI, DCO,  HTML, Javascript, MySQL, {\LaTeX}, GitHub.
    \item \textbf{Pursuits} \textemdash Machine Learning (Deep Learning, Neural Networks, Neural Operators, Computer Vision, Data Analytics), Phase-field modeling, Continuum modeling, Automatic Differentiation, Parallel Computing, Fast Iterative Solvers
  \end{rSection}
  
  \begin{rSection}{Communication} \itemsep -4pt
    \item \bibentry{kapoor2022surrogate}
    \item \bibentry{mlss2022}
    \item \bibentry{cote2021}
  \end{rSection}
  
  \begin{rSection}{Projects} \itemsep -4pt
    \item {\bf Phase field modeling of chemomechanical binary system} \hfill {\em 2022.04 - 2022.08}\\
    {\em Material Mechanics, RWTH Aachen}\\
    Implemented Cahn-Hilliard and Allen-Cahn models to simulate precipitate growth dynamics under the influence of chemical and mechanical energies. (Python) 
    \item {\bf Detecting gravity waves in atmospheric temperature data} \hfill {\em2022.06} (excursion week)\\
    {\em Applied and Computational Mathematics, RWTH Aachen}\\
    Developed a low time-complexity algorithm to detect gravity wave events for reliable weather predictions. Supervised by Dr. Joern Ungermann from FZ Jülich. (Python with NumPy, SciPy, JAX libraries)
    \item {\bf Tracking local optima in dynamic systems}\hfill {\em 2021.10 - 2022.02}\\
    {\em Software and Tools for Computational Engineering, RWTH Aachen}\\
    Developed local-optima-tracking software for dynamic time-dependent functions. Supervised by Prof. Uwe Naumann. (C++ with dco/c++ library)
    \item {\bf Fast iterative solvers for linear systems}\hfill {\em 2021.04 - 2021.09}\\
    {\em AICES, RWTH Aachen}\\
    Implemented multigrid solvers, Krylov-based linear system solvers (GMRES and CG) and eigensolver algorithms (Lanczos and Power Iteration) for huge sparse matrices taken from MatrixMarket. (Python)
    \item {\bf Simulation of mold filling in LPIM (MITACS Scholar)} \hfill {\em 2019.05 - 2019.08}\\
    {\em Département de génie mécanique, ETS Montreal}\\
    Worked in modeling of injection stage in low-pressure metallic-powder injection molding using FEM simulations and experimentation. Supervised by Prof. Vincent Demers. (Moldflow, AutoCAD, MATLAB)
    \item {\bf Phase field modeling of ternary system} \hfill {\em 2018.11 - 2019.04}\\
    {\em Metallurgical and Materials Engineering, NIT Warangal}\\
    Developed a semi-implicit spectral PFM routine to study the growth kinetics of precipitates in a hypothetical ternary system. (C with FFT library)
  \end{rSection}
  

  \begin{rSection}{Volunteering} \itemsep -4pt 
    \item {\bf Project Aakaar: Making geometry accessible to visually impaired}\hfill {\em 2019.01 - Present}\\
    Started as a bachelor's project at the maker space of NIT Warangal, Project Aakaar has expanded into a global network of designers, thinkers, managers, and engineers, with the common goal of making technical education accessible in the special schools of developing countries. In collaboration with Dr. Kyle Keane (MIT).
  \end{rSection}

  \begin{rSection}{Awards} \itemsep -4pt 
    \item Recipient of a full scholarship to attend Machine Learning Summer School 2022
    \item Recipient of Late Pendyala Upendra Gold Medal 2020 for academic excellence in BTech degree
    % \item Honored by Indian Institute of Metals, Hyderabad chapter, for academic excellence in BTech degree
    \item Recipient of MITACS 2019 scholarship to pursue research at ETS Montreal \hfill 
    \item Recipient of NIT Warangal Merit Scholarship 2016, 2017, 2018, 2019 (Full Tuition Award)
    \item Recipient of the prestigious OPJEMS award for two consecutive years 2018, 2019 \hfill 
    % \item Recipient of Central Scheme National Scholarship 2016, 2017, 2018 \hfill 
  \end{rSection}
\end{document}
